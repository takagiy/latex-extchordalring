\documentclass{article}
\usepackage[margin=1in]{geometry}
\usepackage[T1]{fontenc}
\usepackage{lmodern}
\usepackage{biolinum}
\usepackage[noprefix]{extchordalring}
\usepackage{tikz}
\usepackage{showexpl}
\usepackage{color}
\usepackage{caption}

\title{\topskip0pt\vspace*{\fill}\biolinum\bfseries\large Package \\ \Huge extchordalring}
\author{Yuki Takagi \\ \normalsize takagiy.dev@gmail.com}
\date{\vspace*{\fill}}

\lstset{
  frame=tb,
  basicstyle=\ttfamily\bfseries,
  commentstyle=\color{darkgray},
  keywordstyle=\color{blue},
  language=[LaTeX]{TeX}
}

\mathchardef\mhyphen="2D

\begin{document}
\begin{titlepage}
  \maketitle
  \begin{table*}[b!]
    \tableofcontents
  \end{table*}
  \thispagestyle{empty}
\end{titlepage}

\section{Getting started}

You need to load the packages \lstinline{extchordalring} and \lstinline{tikz}
in preambles to use this package.

\begin{lstlisting}
\usepackage{extchordalring}
\usepackage{tikz}
\end{lstlisting}

\section{Standard rings}

The command \lstinline{\xcrCR} is available within the environment \lstinline{tikzpicture}
to draw a chordal ring or an extended, N-chordal ring.

\begin{LTXexample}[pos=r]
\begin{tikzpicture}
  \xcrCR(14, 3)  
\end{tikzpicture}  
\end{LTXexample}

\begin{LTXexample}[pos=r]
% 2-chordal ring CR(14, 3, 7)
\begin{tikzpicture}
  \xcrCR(14, 3, 7)
\end{tikzpicture}
\end{LTXexample}

\section{Drawing edges manually}

Optionally, you can manually specify which edges should appear in the ring.

Space-separated lists or space-separated pairs can be used to draw paths or edges in rings.

\begin{LTXexample}[pos=r]
\begin{tikzpicture}
  \xcrCR(14)[
    0 1 2 3,
    6 5 2 9 10 11
  ]
\end{tikzpicture}  
\end{LTXexample}

You can also use the syntax $\langle a\ vertex\rangle$ \lstinline{adj} $\langle a\ space\mhyphen separated\ list\ of\ vertices\rangle$ to draw edges between a vertex and its adjacent vertices.

\begin{LTXexample}[pos=r]
\begin{tikzpicture}
  \xcrCR(14)[
    1 adj 4 8 12
  ]
\end{tikzpicture}
\end{LTXexample}

Furthermore, these notations can be used simultaneously with each other.

\begin{LTXexample}[pos=r]
\begin{tikzpicture}
  \xcrCR(14)[
    0 1 2 3, 5 6, 9 10 11,
    1 adj 4 8 12,
    2 adj 5 9 13,
    4 7
  ]
\end{tikzpicture}
\end{LTXexample}

A ring with no edges can be drawn by passing \lstinline{[]} as option.

\begin{LTXexample}[pos=r]
\begin{tikzpicture}
  \xcrCR(14)[]
\end{tikzpicture}
\end{LTXexample}

\section{Styling and Preferences}

The package provides some optional keys to change the styles or the preferences of drawn chordal rings.

You can set the optional key \lstinline{xcr radius} which is defaulted to 2, to change the radius of
chordal rings.

\begin{LTXexample}[pos=r]
\begin{tikzpicture}
  \xcrCR[xcr radius=3](14, 3)
\end{tikzpicture}
\end{LTXexample}

The key \lstinline{xcr node/style} can be used to style the vertices in chordal rings.

\begin{LTXexample}[pos=r]
\begin{tikzpicture}
  \xcrCR[
    xcr node/.style={
      shape=circle,
      minimum size=14pt,
      draw=black,
      fill=white,
    }
  ](14, 3)
\end{tikzpicture}
\end{LTXexample}

You can also set the options of styling and preferences through something like the command \lstinline{tikzset} or the options of the environment \lstinline{tikzpicture} as you can for other tikz options.


\begin{LTXexample}[pos=r]
\tikzset{xcr radius=15mm}

\begin{tikzpicture}
  \xcrCR(14, 3)
\end{tikzpicture}
\end{LTXexample}

\begin{LTXexample}[pos=r]
\begin{tikzpicture}[
    xcr radius=15mm,
]
  \xcrCR(14, 3)
\end{tikzpicture}
\end{LTXexample}
\end{document}
